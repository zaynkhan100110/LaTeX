\documentclass[11pt]{article}
\usepackage{amsmath, amsfonts, amssymb}

\pagestyle{empty}


\begin{document}

\begin{center}
\Large \LaTeX\ For Beginner

\small BY

\Large ZAYN KHAN

\vspace{1cm}

\large ( Math Time )
\end{center}



 Ok ! Let's do some math ?
 
This is inline math mode mean doing math with text in a single line 

${(a+b)^2=(a+b)+(a+b)}$ for this just use opening \$\ and closing \$\

(dollar) sign like this with a carrot \^\ (\$\ (a+b)\^\ 2=(a+b)+(a+b) \$\ ) sym.
 
 \vspace{1cm}
 
 This is display math mode mean doing math with text but math will 
 
 show up in a different central line i mean like this

$${(a+b)^2=(a+b)+(a+b)}$$ for this just use opening \$\ and closing \$\ (dollar) sign but need to use (double)

like this (double \$\ (a+b)\^\ 2=(a+b)+(a+b) double \$\ )

\vspace{1cm}

Here is one more but little different from before
 $$(a-b)_2=(a-b)+(a-b)$$
For this i change the \^\ ( carrot sym. ) to \_\ ( underscore ) and ( + ) to ( - )


like this (double \$\ (a-b)\_\ 2=(a-b)+(a-b) double \$\ )


 \pagebreak
 
 
 superscript
 $$x^2$$ $$x^16$$
 $$x^{16}$$
 $$x^{1332123}$$
 
 subscript
 $$x_2$$ $$x_16$$
 $$x_{16}$$
 $$x_{1{3{3{2{1{2{3}}}}}}}$$
 $$x_{1{3{3{2{1}}2}3}}$$
 $$x_{1332123}$$
 
 explanation ::
 
 superscript ( \^\ )::::::::::::::::
 
 remember double \$\ will automatically created a extra line without
 
 giving that.
 
1.
 
  For superscript after double \$\ sign in 1st one i use 
 
 x then a carrot sym. \^\ then number (double \$\ x \^\ 2 double \$\ ) 
 
2. 
 
 but when use i same method in 2nd on then that is not actual what we
 
 need. I mean 1 is get in superscript but 6 is not. This 2 proves that
 
 double \$\ doesn't need to give line space manually cuz i put them
 
 in 1 line.
 
 3.
 
 And the 3rd one we do same but little customization like put 16 in \{\}
 
 like this ( double \$\ x \^\ \{ 16 \} double \$\ )
 
 
4.
 
 And the 4th one is similar to 3rd one if you want put multiple 
 
 numbers in superscript then just use \{\} like this method
 
 ( double \$\ x \^\ \{ 1332123 \} double \$\ )

\pagebreak
 explanation ::

 subscript ( \_\ )::::::::::::::::
 
 1.
 
  For subscript after double \$\ sign in 1st one i use 
 
 x then a underscore \_\ then number (double \$\ x \_\ 2 double \$\ ) 
 
2. 
 
 but when i use same method in 2nd on then that is not actual what we
 
 need. I mean 1 is get in subscript but 6 is not.
 
 3.
 
  And the 3rd one we do same but little customization like put 16 in \{\}
 
 like this ( double \$\ x \_\ \{ 16 \} double \$\ ) 
 
 4.
 
 Here a are more modifications for confusing people who gonna read your 
 
 code give one \{\} for one or each number like this more give blur on eyes
 
 ( double\$\ x\_\ \{1\{3\{3\{2\{1\{2\{3 \}\}\}\}\}\}\} double \$\ )
 
 5.
 
 In the result of 4th and 5th can be same but for input we can make some 
 
 changes like ( double \$\ x\_\ \{1\{3\{3\{2\{1\}\}2\}3\} double \$\ )
 
 \vspace{2cm}
 
 6.
 
 This time is normal and similar to superscript if you want put
 
  multiple numbers in subscript then just use \{\} like this method
 
 ( double \$\ x \_\ \{ 1332123 \} double \$\ )
 
 \vspace{2cm}
 
  explanation ::

 Remember ( \^\  \_\ )::::::::::::::::
 
 I use () before \$\ and after \$\ just for make this more readable but don't
 
 use () everywhere in \TeX\ file and also avoid all space cuz there are no
 
any space between those numbers this space is for just good explanation.

\pagebreak

Time for roots ! I mean  $\sqrt{2}$ this kind of stuff
$$\sqrt{2}$$
For do this just use ( double \$\ $\backslash$sqrt\{2\} double \$\ )
$$\sqrt[5]{2}$$
For this just use ( double \$\ $\backslash$sqrt[5]\{2\} double \$\ )
$$\sqrt{a+b^2}$$
For do this just use ( double \$\ $\backslash$sqrt\{a+b \^\ 2\} double \$\ )
$$\sqrt[a+b^6]{a-b_1}$$
For this one just use ( double \$\ $\backslash$sqrt\ [a+b\^\ 6]\{a-b\_\ 1\} double \$\ )
$$\sqrt{5+\sqrt{6}}$$
For this just use ( double \$\ $\backslash$sqrt\{5+$\backslash$sqrt\{6\}\} double \$\ )

1. 
$$\sqrt[a+b^3]{(a+b_2)+\sqrt{(a+b)}}$$

2.
$$\sqrt[(a-b)_10]{a-b^8-\sqrt[r(c-d)^8]{(c-d)_{10}}}$$

Exam Time time to test what you get :::: 

Explain how i print 1 and 2. I don't give any hint but you can

find how to cook. Recipe is on page 1 to 4.

Tips ::::

Push yourself to read and understand again.

After your final push fell free to peak at \TeX\ file.

But think in simple ways cuz this cannot be found with hard way.

\pagebreak

Let's do some fractions

$$\frac{1}{2}$$

Just type ( double \$\ $\backslash$frac\{1\}\{2\} double \$\ )

$$\frac{10+5}{2+13}$$

Type ( double \$\ $\backslash$frac\{10+5\}\{2+13\} double \$\ )

$$\frac{10}{2+\frac{5}{x}}$$

This ( double \$\ $\backslash$frac\{10\}\{2+$\backslash$frac \{5\}\{x\}\} double \$\ )

$$\frac{x}{y}+{2}{3}$$

Use ( double \$\ $\backslash$frac\{x\}\{y\}+\{2\}\{3\} double \$\ )

$$\frac{A+b+C^2}{a+B+C_5}$$

Here ( double \$\ $\backslash$frac\{A+b+C\^\ 2\}\{a+B+C\_\ 5\} double \$\ )
$$\frac{\sqrt{5}}{\sqrt{6}}$$

Yeah you can add root in frac just do like this  

{( double \$\ $\backslash$frac\{$\backslash$sqrt\{5\}\} top

             \{$\backslash$sqrt\{6\}\} double \$\ )} bottom
 $$\frac{a+b^4\sqrt{ab_2}}{a-b_2\sqrt{ab^4}}$$
 
 Do ( double \$\ frac\{a+b\^\ 4 $\backslash$sqrt\{a-b\_2\ \}\} top
 
          \{a-b\_\ 2 $\backslash$sqrt\{ab\^\ 4\}\} double \$\ ) bottom
          
          
\vspace{1cm}

Tips ::::

Frac or fraction can remember or do fraction just first 2 digits or letter.

Frack or fraction doesn't character orientation means he doesn't care

about small or capital letters.

Do now more ground with frac or fraction.
\pagebreak

Ok ! Let's have some brackets for advance work.

$$\{\frac{a+b^5}{a-b_5}\}$$

This is not good enough or what we need but this can do by

( double\$\ $\backslash$frac\{a+b\^\ 5\}\{a-b\_\ 5\} double \$\ $\backslash$ )

$$\left\{\frac{a+b^5}{a-b_5}\right\}$$

But this one is looking good or what we need so for do this

( $\backslash$left\{ double\$\ $\backslash$frac\{a+b\^\ 5\} top

\{a-b\_\ 5\}  $\backslash$right\} double \$\ ) bottom

$$ \left\{ \frac{a+b^5}{a-b_5} + \frac{\sqrt{a-b^5}}{\sqrt{a+b_5}}  \right\}$$

If you read all others before then this will be piece of cake


( $\backslash$left\{ double\$\ $\backslash$frac\{a+b\^\ 5\} top left

\{a-b\_\ 5\}+ bottom left 

$\backslash$frac\{ $\backslash$sqrt\{a-b\^\ 5\}\} top right

\{backslash sqrt\{a+b\_\ 5\}\} $\backslash$right\}double \$\ ) bottom right

$$\left\{\frac{a+\frac{5}{10}}{b-\frac{2}{4}}\right\}$$

This is quite good ? Nah ! good will come after this. For this type 

( double\$\ $\backslash$left\{backslash frac\{a+ $\backslash$frac \{5\}\{10\}\}

\{b- $\backslash$frac \{2\}\{4\}\} $\backslash$right\} double \$\ )

$$\left\{\frac{a^2+\frac{5}{10}}{b_8-\frac{2}{4}}+  \frac{\sqrt{b+\frac{2}{4}_8}}
{\sqrt{a-\frac{5}{10}^2}}  \right\}$$

So this can be rude for you if you don't take previous notes. Ok ! Then

( double \$\ $\backslash$left\{ $\backslash$frac\{a\^\ 2+backslash frac\{5\}\{10\}\}

\{b\_\ 8- $\backslash$frac\{2\}\{4\}+ 

(end of first fraction.)

$\backslash$frac\{backslash sqrt\{b+$\backslash$frac\{2\}\{4\}\_\ 8\}\}

\{$\backslash$sqrt\{a-$\backslash$frac\{5\}\{10\}\^\ 2\}\}$\backslash$right\} double\$\ )

(end of last fraction.)

$$\left\{\frac{a^2+\frac{5}{10}}{b_8-\frac{2}{4}}  \right\} + \left\{  \frac{\sqrt{b+\frac{2}{4}_8}}{\sqrt{a-\frac{5}{10}^2}}  \right\}$$

So ! This one is almost similar to the previous one just need to add

   $\backslash$right\} in (end of first fraction.) the + is between extra brackets
   
or middle of those two fraction. $\backslash$left\{ (start of last fraction).

SO time for brainwash :::::

$$100\div2+\left[ 20+10-\left\{5+9-\left(2\times2\right)+5\right\}+10\right]\times2$$
$$100\div2+\left[ 20+10-\left\{5+9-4+5\right\}+10\right]\times2$$
$$100\div2+\left[ 20+10-\left\{5+5+5\right\}+10\right]\times2$$
$$100\div2+\left[ 20+10-15+10\right]\times2$$
$$100\div2+\left[ 20+-5+10\right]\times2$$
$$100\div2+\left[ 15+10\right]\times2$$
$$100\div2+25\times2$$
$$50+25\times2$$
$$50+50$$
$$100$$
This need so find out by yourself use your fresh brain again think easy.
\vspace{1cm}

Tips ::::

If you want to put brackets in normal way like (20)then this can good 

for just single number or inline numbers but when it comes to frac , sqrt 

where numbers can be part then this is not a good choice so for part 

numbers or command like frac , sqrt we can use ( $\backslash$right your 

brackets in begin and $\backslash$right at the end of command).




\end{document}