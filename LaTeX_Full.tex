
\documentclass[11pt]{article}
\usepackage{float}
\usepackage{amsmath, amsfonts, amssymb}
\usepackage{enumerate}
\usepackage{url}
\usepackage{hyperref}
\pagestyle{empty}


\begin{document}

\begin{center}
\Large \LaTeX\ For Beginner

\small BY

\Large ZAYN KHAN
\end{center}

 Hey ! Here i am gonna teach you some basic \TeX\ commands. Let's go.
 

 This will be go on 2nd line for do this you need to to indented this to next line by giving a extra line space between 2 lines otherwise this will be looks like this 4 lines because tex will detect this in 1 paragraph or line remember \LaTeX\ or \TeX\ doesn't provide next line automatically
 
  \vspace{1cm}
  
 And for give more custom space use $\backslash$vspace\{1cm\} after 1
 
 line of blank space (1 c.m. for 1 c.m. of line space between lines )
 
 \vspace{1cm}
 Ok ! Let's do some math ?
 
This is inline math mode mean doing math with text in a single line 

${(a+b)^2=(a+b)+(a+b)}$ for this just use opening \$\ and closing \$\

(dollar) sign like this with a carrot \^\ (\$\ (a+b)\^\ 2=(a+b)+(a+b) \$\ ) sym.
 
 \vspace{1cm}
 
 This is display math mode mean doing math with text but math will 
 
 show up in a different central line i mean like this

$${(a+b)^2=(a+b)+(a+b)}$$ for this just use opening \$\ and closing \$\ (dollar) sign but need to use (double)

like this (double \$\ (a+b)\^\ 2=(a+b)+(a+b) double \$\ )

\vspace{1cm}

Here is one more but little different from before
 $$(a-b)_2=(a-b)+(a-b)$$
For this i change the \^\ ( carrot sym. ) to \_\ ( underscore ) and ( + ) to ( - )


like this (double \$\ (a-b)\_\ 2=(a-b)+(a-b) double \$\ )


 \pagebreak
 
 
 superscript
 $$x^2$$ $$x^16$$
 $$x^{16}$$
 $$x^{1332123}$$
 
 subscript
 $$x_2$$ $$x_16$$
 $$x_{16}$$
 $$x_{1{3{3{2{1{2{3}}}}}}}$$
 $$x_{1{3{3{2{1}}2}3}}$$
 $$x_{1332123}$$
 
 explanation ::
 
 superscript ( \^\ )::::::::::::::::
 
 remember double \$\ will automatically created a extra line without
 
 giving that.
 
1.
 
  For superscript after double \$\ sign in 1st one i use 
 
 x then a carrot sym. \^\ then number (double \$\ x \^\ 2 double \$\ ) 
 
2. 
 
 but when use i same method in 2nd on then that is not actual what we
 
 need. I mean 1 is get in superscript but 6 is not. This 2 proves that
 
 double \$\ doesn't need to give line space manually cuz i put them
 
 in 1 line.
 
 3.
 
 And the 3rd one we do same but little customization like put 16 in \{\}
 
 like this ( double \$\ x \^\ \{ 16 \} double \$\ )
 
 
4.
 
 And the 4th one is similar to 3rd one if you want put multiple 
 
 numbers in superscript then just use \{\} like this method
 
 ( double \$\ x \^\ \{ 1332123 \} double \$\ )

\pagebreak
 explanation ::

 subscript ( \_\ )::::::::::::::::
 
 1.
 
  For subscript after double \$\ sign in 1st one i use 
 
 x then a underscore \_\ then number (double \$\ x \_\ 2 double \$\ ) 
 
2. 
 
 but when i use same method in 2nd on then that is not actual what we
 
 need. I mean 1 is get in subscript but 6 is not.
 
 3.
 
  And the 3rd one we do same but little customization like put 16 in \{\}
 
 like this ( double \$\ x \_\ \{ 16 \} double \$\ ) 
 
 4.
 
 Here a are more modifications for confusing people who gonna read your 
 
 code give one \{\} for one or each number like this more give blur on eyes
 
 ( double\$\ x\_\ \{1\{3\{3\{2\{1\{2\{3 \}\}\}\}\}\}\} double \$\ )
 
 5.
 
 In the result of 4th and 5th can be same but for input we can make some 
 
 changes like ( double \$\ x\_\ \{1\{3\{3\{2\{1\}\}2\}3\} double \$\ )
 
 \vspace{2cm}
 
 6.
 
 This time is normal and similar to superscript if you want put
 
  multiple numbers in subscript then just use \{\} like this method
 
 ( double \$\ x \_\ \{ 1332123 \} double \$\ )
 
 \vspace{2cm}
 
  explanation ::

 Remember ( \^\  \_\ )::::::::::::::::
 
 I use () before \$\ and after \$\ just for make this more readable but don't
 
 use () everywhere in \TeX\ file and also avoid all space cuz there are no
 
any space between those numbers this space is for just good explanation.

\pagebreak

Time for roots ! I mean  $\sqrt{2}$ this kind of stuff
$$\sqrt{2}$$
For do this just use ( double \$\ $\backslash$sqrt\{2\} double \$\ )
$$\sqrt[5]{2}$$
For this just use ( double \$\ $\backslash$sqrt[5]\{2\} double \$\ )
$$\sqrt{a+b^2}$$
For do this just use ( double \$\ $\backslash$sqrt\{a+b \^\ 2\} double \$\ )
$$\sqrt[a+b^6]{a-b_1}$$
For this one just use ( double \$\ $\backslash$sqrt\ [a+b\^\ 6]\{a-b\_\ 1\} double \$\ )
$$\sqrt{5+\sqrt{6}}$$
For this just use ( double \$\ $\backslash$sqrt\{5+$\backslash$sqrt\{6\}\} double \$\ )

1. 
$$\sqrt[a+b^3]{(a+b_2)+\sqrt{(a+b)}}$$

2.
$$\sqrt[(a-b)_10]{a-b^8-\sqrt[r(c-d)^8]{(c-d)_{10}}}$$

Exam Time time to test what you get :::: 

Explain how i print 1 and 2. I don't give any hint but you can

find how to cook. Recipe is on page 1 to 4.

Tips ::::

Push yourself to read and understand again.

After your final push fell free to peak at \TeX\ file.

But think in simple ways cuz this cannot be found with hard way.

\pagebreak

Let's do some fractions

$$\frac{1}{2}$$

Just type ( double \$\ $\backslash$frac\{1\}\{2\} double \$\ )

$$\frac{10+5}{2+13}$$

Type ( double \$\ $\backslash$frac\{10+5\}\{2+13\} double \$\ )

$$\frac{10}{2+\frac{5}{x}}$$

This ( double \$\ $\backslash$frac\{10\}\{2+$\backslash$frac \{5\}\{x\}\} double \$\ )

$$\frac{x}{y}+{2}{3}$$

Use ( double \$\ $\backslash$frac\{x\}\{y\}+\{2\}\{3\} double \$\ )

$$\frac{A+b+C^2}{a+B+C_5}$$

Here ( double \$\ $\backslash$frac\{A+b+C\^\ 2\}\{a+B+C\_\ 5\} double \$\ )
$$\frac{\sqrt{5}}{\sqrt{6}}$$

Yeah you can add root in frac just do like this  

{( double \$\ $\backslash$frac\{$\backslash$sqrt\{5\}\} top

             \{$\backslash$sqrt\{6\}\} double \$\ )} bottom
 $$\frac{a+b^4\sqrt{ab_2}}{a-b_2\sqrt{ab^4}}$$
 
 Do ( double \$\ frac\{a+b\^\ 4 $\backslash$sqrt\{a-b\_2\ \}\} top
 
          \{a-b\_\ 2 $\backslash$sqrt\{ab\^\ 4\}\} double \$\ ) bottom
          
          
\vspace{1cm}

Tips ::::

Frac or fraction can remember or do fraction just first 2 digits or letter.

Frack or fraction doesn't character orientation means he doesn't care

about small or capital letters.

Do now more ground with frac or fraction.
\pagebreak

Ok ! Let's have some brackets for advance work.

$$\{\frac{a+b^5}{a-b_5}\}$$

This is not good enough or what we need but this can do by

( double\$\ $\backslash$frac\{a+b\^\ 5\}\{a-b\_\ 5\} double \$\ $\backslash$ )

$$\left\{\frac{a+b^5}{a-b_5}\right\}$$

But this one is looking good or what we need so for do this

( $\backslash$left\{ double\$\ $\backslash$frac\{a+b\^\ 5\} top

\{a-b\_\ 5\}  $\backslash$right\} double \$\ ) bottom

$$ \left\{ \frac{a+b^5}{a-b_5} + \frac{\sqrt{a-b^5}}{\sqrt{a+b_5}}  \right\}$$

If you read all others before then this will be piece of cake


( $\backslash$left\{ double\$\ $\backslash$frac\{a+b\^\ 5\} top left

\{a-b\_\ 5\}+ bottom left 

$\backslash$frac\{ $\backslash$sqrt\{a-b\^\ 5\}\} top right

\{$\backslash$sqrt\{a+b\_\ 5\}\} $\backslash$right\}double \$\ ) bottom right

$$\left\{\frac{a+\frac{5}{10}}{b-\frac{2}{4}}\right\}$$

This is quite good ? Nah ! good will come after this. For this type 

( double\$\ $\backslash$left\{$\backslash$frac\{a+ $\backslash$frac \{5\}\{10\}\}

\{b- $\backslash$frac \{2\}\{4\}\} $\backslash$right\} double \$\ )

$$\left\{\frac{a^2+\frac{5}{10}}{b_8-\frac{2}{4}}+  \frac{\sqrt{b+\frac{2}{4}_8}}
{\sqrt{a-\frac{5}{10}^2}}  \right\}$$

So this can be rude for you if you don't take previous notes. Ok ! Then

( double \$\ $\backslash$left\{ $\backslash$frac\{a\^\ 2+$\backslash$frac\{5\}\{10\}\}

\{b\_\ 8- $\backslash$frac\{2\}\{4\}+ 

(end of first fraction.)

$\backslash$frac\{$\backslash$sqrt\{b+$\backslash$frac\{2\}\{4\}\_\ 8\}\}

\{$\backslash$sqrt\{a-$\backslash$frac\{5\}\{10\}\^\ 2\}\}$\backslash$right\} double\$\ )

(end of last fraction.)

$$\left\{\frac{a^2+\frac{5}{10}}{b_8-\frac{2}{4}}  \right\} + \left\{  \frac{\sqrt{b+\frac{2}{4}_8}}{\sqrt{a-\frac{5}{10}^2}}  \right\}$$

So ! This one is almost similar to the previous one just need to add

   $\backslash$right\} in (end of first fraction.) the + is between extra brackets
   
or middle of those two fraction. $\backslash$left\{ (start of last fraction).

SO time for brainwash :::::

$$100\div2+\left[ 20+10-\left\{5+9-\left(2\times2\right)+5\right\}+10\right]\times2$$
$$100\div2+\left[ 20+10-\left\{5+9-4+5\right\}+10\right]\times2$$
$$100\div2+\left[ 20+10-\left\{5+5+5\right\}+10\right]\times2$$
$$100\div2+\left[ 20+10-15+10\right]\times2$$
$$100\div2+\left[ 20+-5+10\right]\times2$$
$$100\div2+\left[ 15+10\right]\times2$$
$$100\div2+25\times2$$
$$50+25\times2$$
$$50+50$$
$$100$$
This need so find out by yourself use your fresh brain again think easy.
\vspace{1cm}

Tips ::::

If you want to put brackets in normal way like (20)then this can good 

for just single number or inline numbers but when it comes to frac , sqrt 

where numbers can be part then this is not a good choice so for part 

numbers or command like frac , sqrt we can use ( $\backslash$right your 

brackets in begin and $\backslash$right at the end of command).

\pagebreak

Let's make tables !
\vspace{1cm}

\begin{tabular}{|c||c|c|c|c|c|c|}
\hline
$Name$ & Kabir & Amir & Jim & Afridi & Towhid \\ \hline
$Marks$ & 34.4 & 35.7 & 33.6 & 36.3 & 34.8 \\ \hline

\end{tabular}
\vspace{1cm}

For this type or do. Remember this $ | $ decide box end how many box you 

want just type like this $\backslash$hline \{$|$c$|$\} $\backslash$hline for one box

and \& decide words end or one box words end

and $\backslash$hline decide next row and c for center row align

( $\backslash$begin\{ tabular\} \{ $|$c$|$c$|$c$|$c$|$c$|$c$|$ \} $\backslash$hline 

\$\ Name \$\ \& Kabir \& Amir \& Jim \& Afridi \& Towhid double $\backslash$

 $\backslash$hline )



\vspace{1cm}

\begin{tabular}{|c||c|c|c|c|c|c|}
\hline

$Owner$ & Kabir & Amir & Jim & Afridi & Towhid \\ \hline
$Space$ & $\frac{1}{5}$ & $\frac{1}{5}$ & $\frac{1}{5}$ & $\frac{1}{5}$ & $\frac{1}{5}$ \\ \hline

\end{tabular}

\vspace{1cm}


This is same as before just changed the value to fraction value or frac

Let's do something scary but easy !



\vspace{1cm}

\begin{tabular}{|c|c|c|c|c|c|c|}
\hline
$NAME$ & $CLASS$ & $SEAT NO.$ & $MARKS$ & $RESULT$ & $RANK$ & $YEAR$ \\ \hline
Mustofa & $XII$ &  27492 & 810 & Pass & $1^{st}$ & 2025 \\ \hline
Ali & $XII$ &  23422 & 780 & Pass & $2^{nd}$ & 2025 \\ \hline
Hamar & $XII$ &  27345 & 770 & Pass & $3^{rd}$ & 2025 \\ \hline
Kazi & $XII$ &  25948 & 745 & Pass & $4^{th}$ & 2025 \\ \hline
Faraz & $XII$ &  27965 & 740 & Pass & $5^{th}$ & 2025 \\ \hline
Zakir & $XII$ &  29430 & 737 & Pass & $6^{th}$ & 2025 \\ \hline
Maruf & $XII$ &  26403 & 722 & Pass & $7^{th}$ & 2025 \\ \hline
Afzal & $XII$ &  28204 & 726 & Pass & $8^{th}$ & 2025 \\ \hline
Kaser & $XII$ &  29806 & 720 & Pass & $9^{th}$ & 2025 \\ \hline
Faysal & $XII$ &  20729 & 716 & Pass & $10^{th}$ & 2025 \\ \hline



\end{tabular}

\vspace{1cm}

 This is a normal result sheet of class 12. Find out how i do this.
 
 \pagebreak

\begin{tabular}{|c|l|p{5cm}|}
\hline
NAME & MAIL & YOUR FEEDBACK \\ \hline

Hamar & $hamar9@mail.com$ & If you want to make like this then follow this \\ \hline
Zakir & $zakir8@mail.org$ & $\backslash$begin\{tabular\}\{c$|$l$|$p\{5cm\} $\backslash$hline \\ \hline
Faysal & $faysal7@mail.io$ & this is same as before but just for i need to enter big data in one frame so that's why i choose section {p} in the begin after begin tabular p refers to paragraph and {5cm} refers to a row size in width but how much you write this will increasing his size in height and before p we also have new value called {l} l refers to left align remember c refers to center and all things are same as well ohh ! one more thing if you put {r} it will refers to right \\ \hline
\end{tabular}

\vspace{1cm}

Read that. That will refers all things about this.

\pagebreak
Let's find some align
\begin{align}
(a+b)^2=(a+b)(a+b)\\
(a-b)_2=(a-b)(a-b)
\end{align}
Ok this can be seem to be same with the math mode. But this is not let's

 break down

( $\backslash$begin\{align\} 

(a+b)\^\ 2=(a+b)(a+b) double $\backslash$ 

$\backslash$end\{align\}

see now we don't need to double\$\ or goto math mode and automatically

 get
line numbers. This can help you to locate lines or specify lines.

\begin{align*}
(a+b)^2=(a+b)(a+b)\\
(a-b)_2=(a-b)(a-b)
\end{align*}

In cases if you don't want to show your line numbers then you can do

 this

in the begin and end ( $\backslash$begin\{align\} and $\backslash$end\{align\}

after word align just add * ( asterisk ) this will forbidden line number )

\begin{align*}
&=100\div2+[20+10-\{5+9-(2\times2)+5\}+10]\times2
\end{align*}

In cases if you need or want  ( equals) from beginning then do this

( $\backslash$begin\{align\}

\& =100 $\backslash$div2+[20+10-$\backslash$\{5+9-(2 $\backslash$times2)+$\backslash$\}+10]

$\backslash$times2 

$\backslash$end \{align\} )

you cannot normaly add equal sign = in align so \& helps you to get that

\pagebreak

Time for mess :::

\begin{align*}
&=100\div2+[20+10-\{5+9-(2\times2)+5\}+10]\times2 \\
&=100\div2+[20+10-\{5+9-4+5\}+10]\times2 \\
&=100\div2+[20+10-\{5+5+5\}+10]\times2 \\
&=100\div2+[20+10-15+10]\times2 \\
&=100\div2+[20+-5+10]\times2 \\
&=100\div2+[15+10]\times2 \\
&=100\div2+25\times2 \\
&=50+25\times2 \\
&=50+50 \\
&=100 \\
\end{align*}
The answer is 100.

Find how i made this by your own.

Tips ::::

Remember don't give a line space or don't write anything form begin to 

end .

\pagebreak



Let's do some other kind of lines !

\begin{enumerate}
\item Dates \item Orange \item Apple
\end{enumerate}

Here how to do this.

$\backslash$begin\{enumerate\}

$\backslash$item Dates
$\backslash$item Orange
$\backslash$item Apple

$\backslash$end\{enumerate\}

Also like this

$\backslash$begin\{enumerate\}

$\backslash$item Dates

$\backslash$item Orange

$\backslash$item Apple

$\backslash$end\{enumerate\}

Let's try with bullet style !

\begin{itemize}
\item Dates \item Orange \item Apple
\end{itemize}

Here.

$\backslash$begin\{itemize\}

$\backslash$item Dates

$\backslash$item Orange

$\backslash$item Apple

$\backslash$end\{itemize\}


\begin{enumerate}[A.]
\item Dates
\item Orange 
\item Apple
\end{enumerate}

For giving Letters for lines style you should do this.

$\backslash$begin\{enumerate\}[A.]

$\backslash$item Dates

$\backslash$item Orange

$\backslash$item Apple

$\backslash$end\{enumerate\}

You will need a package for this called \{ enumerate \}

\pagebreak

Till now look every words focus on Start let's do something different !

\begin{enumerate}
\item[Dates are so sweet]
\item[Orange are good water]
\item[Apple isn't for doctor]
\end{enumerate}

See now they focus on end now. Here's how to do it.

$\backslash$begin\{enumerate\}

$\backslash$item [Dates are so sweet]

$\backslash$item [Orange are good water]

$\backslash$item [Apple isn't for doctor]

$\backslash$end\{enumerate\}

something you like \{ maybe \}

\begin{enumerate}
\item Dates
\begin{enumerate}
\item They are so sweet
\end{enumerate}
\end{enumerate}

Yes you can make sub list or sub sub list or more. Here how to do.

$\backslash$begin\{enumerate\}

$\backslash$item Dates

$\backslash$begin\{enumerate\}

$\backslash$item They are so sweet

$\backslash$end\{enumerate\}

$\backslash$end\{enumerate\}

\begin{itemize}
\item Orange
\begin{itemize}
\item Orange are good water
\end{itemize}
\end{itemize}

$\backslash$begin\{itemize\}

$\backslash$item Dates

$\backslash$begin\{itemize\}

$\backslash$item They are so sweet

$\backslash$end\{itemize\}

$\backslash$end\{itemize\}

\pagebreak

\begin{enumerate}[A.]
\item Apple
\begin{enumerate}
\item Apple isn't for doctor
\end{enumerate}
\end{enumerate}

$\backslash$begin\{enumerate\}[A.]

$\backslash$item Apple

$\backslash$begin\{enumerate\}

$\backslash$item Apple isn't for doctor

$\backslash$end\{enumerate\}

$\backslash$end\{enumerate\}

\vspace{1cm}

Let's do some fun !

\begin{enumerate}
\item Dates
\begin{itemize}
\item They are so sweet
\begin{enumerate}[A.]
\item This will give you extra energy
\item[This can eat on winter for get warm]
\end{enumerate}
\end{itemize}
\item Orange
\begin{itemize}
\item Orange are good water
\begin{enumerate}[A.]
\item This have so much vitamin c
\item[You can give someone who is sick]
\end{enumerate}
\end{itemize}
\item Apple
\begin{itemize}
\item Apple isn't for doctor
\begin{enumerate}[A.]
\item Apple are rich fruit
\item[Don't chop them eat by teeth]
\end{enumerate}
\end{itemize}
\end{enumerate}

\pagebreak

\begin{enumerate}
\item [1.] Coconut
\begin{itemize}
\item [1.] Coconut water is good for health
\begin{enumerate}[A.]
\item [1.] Coconut oil is so good
\item [1.] [This have a solid shell]
\end{enumerate}
\end{itemize}
\item [1.] Orange
\begin{itemize}
\item [1.] Orange are good water
\begin{enumerate}[A.]
\item [1.] This have so much vitamin c
\item [1.] [You can give someone who is sick]
\end{enumerate}
\end{itemize}
\item [1.] Apple
\begin{itemize}
\item [1.] Apple isn't for doctor
\begin{enumerate}[A.]
\item [1.] Apple are rich fruit
\item [1.] [Don't chop them eat by teeth]
\end{enumerate}
\end{itemize}
\end{enumerate}


Here i just use [1.] after $\backslash$item

\vspace{1cm}

Tips ::::

If you use custom line code then right align doesn't work.

Find yourself for better understand. This can good for your brain.

\pagebreak

Let's Do Some Extra Function !

If you want to center you letters without mathmode ( double \$\ )

\begin{center}This will shows in center.\end{center} 

You can do this 

 ( $\backslash$begin\{center\} Your words  $\backslash$end\{center\} )
 
 Or This
 
 ( $\backslash$begin\{center\}
 
  Your words 
  
   $\backslash$end\{center\} )

\vspace{1cm}

If you want to align right then

\begin{flushright}
This will show on right
\end{flushright}

\vspace{1cm}

Or if you want to left align then

\begin{flushleft}
This will show on left
\end{flushleft}

Recipe 

( $\backslash$begin\{flushright\}

  Your words here

  $\backslash$end\{flushright\} )
  
  For left just change ( flushright ) to ( flushleft )
  
\vspace{1cm}

If you want to make Text bigger or smaller then do this like :

\begin{enumerate}

\item \tiny{This is called (tiny)}
\item \scriptsize{This is called (scriptsize)}
\item \small{This is called (small)}
\item \normalsize{This is called (normalsize)}
\item \large{This is called (large)}
\item \Large{This is called (Large)}
\item \LARGE{This is called (LARGE)}
\item \huge{This is called (huge)}

\end{enumerate}  
  
  
  \vspace{1cm}  
  
  I use enumerate just for better explain you don't need to use
  
  enumerate while writing an actual document for example like :
  
  \huge{This is called (huge)}

Recipe :

There are two ways to do like this

1. $\backslash$which size you want \{words\}

(e.g. $\backslash$Large\{This is called(Large)\}

\vspace{1cm}

\normalsize Something is wrong right ? From before recipe to e.g everything is in 

huge size but why ? And why not from 7.(huge) ?

The answer is the 7.(huge is inside the enumerate which have begin and

end point that's why that don't flow up to the next documents and

before recipe (huge) we don't have any begin or close point also

i am writing this question and answer using $\backslash$normalsize

\vspace{1cm}

Then why i show this? And is there any help to stop them from flow?

The answer is i show you for save your time and yeah there is way for

stop flow over the next documents. Here is it:

\vspace{1cm}

{\large This is called (large)}

\vspace{1cm}

see now this isn't flow this time. Here is the recipe:

\{$\backslash$large your words\}  . This is just about \{\} and knowledge.

Remember : between $\backslash$and large there is no space but after 

$\backslash$large there need a space like \{$\backslash$large your words\}

\pagebreak

{\large 2.}

So here is the way no. 2

\begin{tiny}
This is called tiny
\end{tiny}

\begin{scriptsize}
This is called scriptsize
\end{scriptsize}

\begin{small}
This is called small
\end{small}

\begin{normalsize}
This is called normal size
\end{normalsize}

\begin{large}
This is called large
\end{large}

\begin{Large}
This is called large
\end{Large}

\begin{LARGE}
This is called LARGE
\end{LARGE}

\begin{huge}
This is called huge
\end{huge}

Recipe :

$\backslash$begin\{your size\}

your words

$\backslash$end\{your size\}

(e.g.)

$\backslash$begin\{large\}

your words

$\backslash$end\{large\}

\vspace{1cm}

End of the day we found 3 ways to do size filtering :

1. enumerate

2. $\backslash$size

3. using begin $\backslash$size 

Remember : enumerate gives you next line automatically but if you using

$\backslash$size or begin $\backslash$size they don't give you you need to

do that by your own.

\vspace{1cm}

enumerate is good for line up while you are learning this can help you

to learn two things in once. You can also use itemize or custom.

\vspace{1cm}

$\backslash$size can help you for time saving when you are in hurry and

this looks cool for me. But use the \{\} begin size and after words for

stop flow.

$\backslash$begin can help for use and located individual 

and looks so rich
also mess up code. This can take your extra time.

\pagebreak

Ok ! Let's try them 4 :

\begin{enumerate}
\item \large{This is written in enumerate}
\end{enumerate}

\large This is written in $\backslash$large without \{\}

\begin{large}
This is written in $\backslash$begin size
\end{large}

{\large This is written in $\backslash$large with \{\} both start and end }

\vspace{1cm}

If you want to make big or small just one word in a sentence then 

do like this :

This is  {\tiny (tiny)} and {\normalsize(normalsize)} or {\huge(huge)}

Recipe write in one line :

This is \{$\backslash$tiny (tiny)\} and 

\{$\backslash$normalsize (normalsize)\}

or \{$\backslash$huge (huge)\}

\vspace{1cm}

\pagebreak

{\Large Time for urls ::: }

If you want to put url on document then do this :

\vspace{1cm}

This is my website  \url{https://zaynkhan100110.github.io}

This is my  \href{https://zaynkhan100110.github.io}{website}

\vspace{1cm}

So i give here 2 type of hyper links which can redirect to the

destination. So here is the recipe ::

1. I give a complete url for my webpage which can take to the link

Destination.

2. I give a link inside of text so if some click on text this will also

take to the destination place ( my webpage ).


Here's how to do ::

\vspace{1cm}

1. This is my website $\backslash$url\{https://zaynkhan100110.github.io\}

\vspace{1cm}

which link i put inside the \{\} that will redirect if you click.

\vspace{1cm}

2. This is my $\backslash$href\{https://zaynkhan100110.github.io\}\{website\}

\vspace{1cm}

so after $\backslash$href \{\} is contain our link for destination and

after link \{\} contain text what will be appear on our document

this one is similar to html href.

\vspace{1cm}

Remember if you want to put links which contain ( \_ ) underscore

or other symbol at that time you need more things to do you can

 see the documentations on overleaf or some other \TeX\ place.

\vspace{1cm}

Note ::

You will need hyperref and url package for this.

\pagebreak

{\large So this page contain some information about page setup/margin so 

let's go ::: }

\vspace{0.5cm}

So as you can see i setup margin of this page 1cm from every side.

I mean ( from top to bottom , right to left ). Here how i do this :::

\vspace{0.5cm}

Before begin document and after document class we need to load a

 package with some attributes ($\backslash$usepackage[margin=1cm]\{geometry\})

So that will do everything about margin i use this now cuz it's quite

 simple in one words. Otherwise need more customize package load 
 
 more specific like if you want to give margin different from everyside
 
  like :::

\vspace{0.5cm}

From bottom you want 0.5cm , top 1cm , right 0.6cm , left 2cm. 

You can also use in for inch.

\vspace{0.5cm}

Do this :::

$\backslash$usepackage[top=1cm, bottom=0.5cm, right=0.6cm, left-2cm]
\{geometry\}

\vspace{0.5cm}

Some paper you can load :::

a0paper, a1paper, a2paper, a3paper, a4paper, a5paper, a6paper,

b0paper, b1paper, b2paper, b3paper, b4paper, b5paper, b6paper,c0paper,

 c1paper, c2paper, c3paper, c4paper, c5paper, c6paper,b0j, b1j,
 
  b2j, b3j, b4j, b5j, b6j,ansiapaper, ansibpaper, ansicpaper,
  
   ansidpaper, ansiepaper,letterpaper, executivepaper, legalpaper

\vspace{0.5cm}

You can set you output paper to landscape with geometry command 

like this :::

$\backslash$usepackage\{geometry\}
$\backslash$geometry\{a4paper, landscape, margin=2in\}

\vspace{0.5cm}

Remember :::

\vspace{0.5cm}

1. LaTeX's margins are, by default, 1.5 inches wide on 

12pt documents, 1.75 inches wide on 11pt documents,

and 1.875 inches wide on 10pt documents.
 
 
2. If you want to give space between words but not line space then 

just press spacebar is not enough for \LaTeX\ you need to give 

$\backslash$,(comma) for one letter of space.



3 .If you want to write \LaTeX\ This way then do this ( $\backslash$LaTeX$\backslash$ )

 second $\backslash$ for one letter of space. Don't try this on another other 
 
 words insted \TeX\ TeX and \LaTeX\ LaTeX and be sure about spelling.

4. Avoid () Those are just for better explanation.

5. I use $\backslash$pagestyle\{empty\} on top after document class for prevent 

printing page number on every page on bottom i don't like that on

 my document avoid this command if you writing a book or want
 
  page number by default \LaTeX\ use page number.

6. For more \TeX\ or \LaTeX\ commands, method use webpage online.

\vspace{10cm}

\begin{center}
\Huge Thank you
\end{center}
\end{document}